a) Since $X$ and $Y$ are independent, $f(x,y)=f(x)f(y) = \dbinom{n}{x}p^{x}(1-p)^{n-x} \lambda e^{-\lambda y}.$\\
$P(X \leq Y) = \sum_{x=0}^{n} \dbinom{n}{x}p^{x}(1-p)^{n-x} \int_{x}^{\infty} \lambda e^{-\lambda y} dy = \sum_{x=0}^{n} \dbinom{n}{x}p^{x}(1-p)^{n-x} e^{-\lambda x} =  \sum_{x=0}^{n} \dbinom{n}{x}(pe^{-\lambda})^{x}(1-p)^{n-x} =(1 - p + pe^{-\lambda})^{n}$\\
\\
b) $P(X_{n} \leq Y_{n}) = (1 - \frac{1}{n} + \frac{1}{n}e^{-\log(2)})^{n} =  (1 - \frac{1}{n} + \frac{1}{2n})^{n} = (1 - \frac{1}{2n})^{n}.$\\
$\sum_{n=1}^{\infty} P(X_{n} \leq Y_{n}) = \sum_{n=1}^{\infty} (1 - \frac{1}{2n})^{n} = \infty$ since $\lim_{n \rightarrow \infty} P(X_{n} \leq Y_{n}) = \lim_{n \rightarrow \infty} (1 - \frac{1}{2n})^{n} = e^{-2} < \infty.$ So, since $X_{n}$ and $Y_{n}$ are all independent, $P(X_{n} \leq Y_{n} \ \ i.o.) = 1$ by the Borel Zero-one law. 